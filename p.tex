
\documentclass[11pt]{article}

\usepackage{amsthm}
\usepackage{indentfirst}
\usepackage{natbib}
\usepackage[colorlinks=true,citecolor=blue,urlcolor=blue]{hyperref}
\usepackage{doi}
\usepackage{url}

\newtheorem{theorem}{Theorem}

\begin{document}

\title{Poisson Approximation}

\author{Charles J. Geyer}

\maketitle

\section{License}

This work is licensed under a Creative Commons
Attribution-ShareAlike 4.0 International License
\url{http://creativecommons.org/licenses/by-sa/4.0/}.

The \LaTeX\ source for this document is in the GitHub
repository \url{https://github.com/cjgeyer/Poisson}.

\section{Introduction}

We develop the theory of Poisson approximation and Poisson point processes
in the context of ``radically elementary probability theory'' \citep{nelson},
which is based on what \citeauthor{nelson} later dubbed ``minimal'' nonstandard
analysis \citep{nelson-minimal}, which is based on four axioms
(\citealp{geyer}, Chapter~2; \citet{nelson}, Chapter~4), which are in turn
theorems of internal set theory \citep{nelson-bull}, which has been proved
to be equiconsistent with ZFC set theory (Zermelo-Frankel with the axiom of
choice).  Thus we are doing mathematics with nonstandard analysis
(infinitesimals and unlimited numbers, which are reciprocals of infinitesimals)
in a completely rigorous way.  For more on this, see \citet[Chapter~2]{geyer}.

We are going to develop using radically elementary probability theory,
the theory of Poisson approximation and Poisson spatial point processes.
We will see --- that as with many aspects of probability theory --- radically
elementary probability theory radically simplifies this subject.  It turns
out that the space in Poisson spatial point processes is irrelevant.
Our main theorem is the following.
\begin{theorem} \label{th:main}
Suppose $T$ is a finite set and $X_t$, $t \in T$ are independent Bernoulli
random variables with infinitesimal expectation.  Then for any $A \subset T$,
$\sum_{t \in A} X_t$ has a nearly Poisson distribution.
\end{theorem}
Nearly Poisson means the probability of any event agrees with that of the
Poisson distribution of conventional mathematics up to near equality
(the difference of the two is infinitesimal).  In order for our theorem
to be correct, we need to allow Poisson random variables with infinitesimal
mean (any random variable which is infinitesimal almost surely) and with
unlimited mean (any random variable which is unlimited almost surely).

Actually, more is true.
\begin{theorem} \label{th:l-p}
With the setup of Theorem~\ref{th:main},
$\sum_{t \in A} X_t$ is $L^p$ for all limited $p$ and moments of all orders
agree with those of the Poisson distribution of conventional probability
theory up to infinitesimals.
\end{theorem}
For more on the concepts infinitesimal almost surely, unlimited almost surely,
and $L^p$, see \citet{geyer}, Chapter 6.

Radically elementary probability theory is based on two principles.
\begin{itemize}
\item All sample spaces are finite.
\item All outcomes have positive probability.
\end{itemize}
One might think this radical simplification of conventional, measure-theoretic
probability theory throws out the baby with the bathwater.  But infinitesimals
and unlimited numbers save the day.  We cannot even define the Poisson
distribution of conventional mathematics if all sample spaces must be finite.
But as our theorems say, we make only infinitesimal errors by truncating
the sample space.

\begin{thebibliography}{}

\bibitem[Geyer(2007)]{geyer}
Geyer, C.~J. (2007).
\newblock Radically Elementary Probability and Statistics.
\newblock Technical Report 657, School of Statistics, University of Minnesota. 
\newblock \url{https://hdl.handle.net/11299/199667}.

\bibitem[Nelson(1977)]{nelson-bull}
Nelson, E. (1977).
\newblock Internal set theory: a new approach to nonstandard analysis.
\newblock \emph{Bulletin of the American Mathematical Society}, 83:1165--1198.
\newblock \doi{10.1090/S0002-9904-1977-14398-X}.

\bibitem[Nelson(1987)]{nelson}
Nelson, E. (1987).
\newblock \emph{Radically Elementary Probability Theory}.
\newblock Princeton University Press.

\bibitem[Nelson(2007)]{nelson-minimal}
Nelson, E. (2007).
\newblock The virtue of simplicity.
\newblock In: van den Berg, I., and Neves, V. (eds.)
    \emph{The Strength of Nonstandard Analysis}.
\newblock Springer, Vienna.
\newblock \doi{10.1007/978-3-211-49905-4\_2}.

\end{thebibliography}

\end{document}

